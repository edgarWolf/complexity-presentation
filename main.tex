\documentclass{beamer}
\usetheme{Boadilla}

\usepackage{ngerman}				% neue deutsche Rechtschreibung
\usepackage[utf8]{inputenc} 		% Umlaute im Text
\usepackage[T1]{fontenc}
\usepackage{xspace}                 % Vermeidung von "ineinanderfallenden f's", wie z.B. bei Schifffahrt
\usepackage{url}		            % korrekte Anzeige/Umbruch von URLs
\usepackage{listings}               % z.B. nützlich zum Einbinden von Quellcode
\usepackage{hyperref} 				% für Hyperlinks in PDF-Dokumenten 
\usepackage{lmodern}
\usepackage{enumerate}
\usepackage[autostyle]{csquotes}
\usepackage{amsthm}
\usepackage{amsmath}
\usepackage{amssymb}

% Bib
\usepackage[backend=biber,style=numeric, citestyle=ieee]{biblatex}
\addbibresource{Literatur/quellen.bib} %Imports bibliography file

\nocite{arora_computational_2009}
\nocite{sipser_introduction_2012}
\nocite{rothe_komplexitatstheorie_2008}
\nocite{aaronson_scott_2016}
\nocite{goos_relativized_1986}
\nocite{rossman_complexity_2015}
\nocite{schaefer_completeness_nodate}

% Grafiken
\usepackage{graphicx} 				% Grafiken einfügen (pdf,png - aber jpg vermeiden)
\graphicspath{{./Bilder/}}          % Pfad zu den Bildern

\title{Polynomialzeithierarchie}
\subtitle{Seminar Komplexität}
\author{Edgar Wolf}
\institute{Hochschule Kempten}
\date{13.07.2022}

\AtBeginSection[]
{
  \begin{frame}
  \tableofcontents[currentsection]
  \end{frame}
}

\theoremstyle{plain}
\newtheorem{satz}{Satz}

\theoremstyle{plain}
\newtheorem{conjecture}{Vermutung}

\begin{document}

\begin{frame}
    \titlepage
\end{frame}

\begin{frame}
    \frametitle{Agenda}
    \tableofcontents
\end{frame}

\section{Einführung}

\begin{frame}
    \frametitle{Einführung}
    Der Nichtdeterminismus dient als ein Konzept, um die Berechnungskraft eines Algorithmus zu erhöhen.
    Jedoch hat auch der Nichtdeterminismus seine Grenzen:
    \begin{itemize}
        \item Probleme aus NP fragen, ob eine Berechnung akzeptiert: \\
         \enquote{Gibt es eine Belegung für die boolsche Formel $F$, sodass $F$ erfüllbar ist?}
        \item Probleme aus coNP fragen, ob alle Berechnungen akzeptieren: \\
          \enquote{Gilt für alle Belegungen für die boolsche Formel $F$, dass $F$ nicht erfüllbar ist?}
    \end{itemize}
    $\rightarrow$ Was tun mit einem Problem, dass beide Bedingungen entscheiden muss?
\end{frame}


 \begin{frame}
    \begin{block}{\textbf{EXACT-INDSET}:}
        \textbf{Eingabe:} Ein Graph $G$, ein $k \in \mathbb{N}$\\ 
        \textbf{Frage:} Hat das größte \textbf{INDSET} in $G$ genau die Größe $k$?
    \end{block}

    Das Problem scheint über kein \enquote{kurzes} Zertifikat zu verfügen. \\
    \textbf{Algorithmus zur Entscheidung des Problems}:
    \begin{itemize}
        \item Rate ein \textbf{INDSET} $U$ mit Größe $k$.
        \item Prüfe, dass es kein \textbf{INDSET} mit einer Größe $l > k$ gibt.
    \end{itemize}
     Wie können wir Probleme solcher Art formal beschreiben?
 \end{frame}
 
 \begin{frame}{Einführung}
    Es gibt Probleme, deren Entscheidbarkeit über den Einsatz nichtdeterministischer Berechnungen hinausgeht.
    Die Beschränkung des Nichtdeterminismus auf die Frage nach der Existenz einer akzeptierenden Berechnung limitiert die Lösungsmöglichkeiten für Probleme,
    die Antworten über mehrere Stufen quantifizieren. \\
    $\rightarrow$ Die Polynomialzeithierarchie bildet diese Probleme ab.
 \end{frame}
\section{Alternierung}

\begin{frame}
    \frametitle{Alternierung}
    Es gibt Probleme, die nicht ausschließlich nichtdeterminstisch gelöst werden können, jedoch immer noch schneller als mit exponentieller Laufzeit.
    Die Beschränkung des Nichtdeterminismus auf die Frage nach der Existenz einer akzeptierenden Berechung limitiert die Lösungsmöglichkeiten für Probleme,
    die Antworten über mehrere Stufen qunatifizieren. Die Alternierung löst dieses Problem.
\end{frame}

\begin{frame}
    \frametitle{Alternierung}
    Alternierung kannn als Verallgmeinerung des Nichtdeterminismus aufgefasst werden, und ist somit ebenfalls ein Konzept, dass in der Praxis kaum umzusetzen ist.
    Während beim Nichtdeterminismus nach einem akzeptierenden Berechnunspfad gesucht wird, können in der Alternierung die Berechnungen quantifiziert werden.
\end{frame}

\subsection{Begriffsklärung}
\begin{frame}
    \frametitle{Alternierung}
    Die Zustände einer Berechnung einer alternierenden Berechnung sind (neben Startzustand und Endzuständen) entweder:
    \begin{itemize}
        \item existenziell: Die Berechnung akzeptiert, wenn mind. eine Berechnung akzeptiert
        \item universell: Die Berechnung akzeptiert, wenn alle Berechnungen akzeptieren.
    \end{itemize}

\end{frame}

\subsection{Alternierende Turingmaschinen}
\begin{frame}
    \frametitle{Alternierende Turingmaschinen}
\end{frame}
\section{Defintionen der PH}

\subsection{Defintion über alternierende Qunatoren}

\subsection{Defintion über alternierende Turingmaschinen}

\subsection{Definition über Turingmaschinen mit Zugriff auf Orakel}
\section{Kollabieren der PH}
\begin{frame}
    \frametitle{Kollabieren der PH}
    Es wird nun auf die Möglichkeit eines Kollabierens der PH eingegangen.
    Da für die Betrachtung dieser Möglichkeit die Vollständigkeit in der PH von zentraler Bedeutung ist, wird zunächst diese formal eingeführt.
\end{frame}

\subsection{Vollständigkeit}
\begin{frame}
    \frametitle{Kollabieren der PH}
    \framesubtitle{Vollständigkeit}
    \begin{block}{\textbf{$\Sigma^p_i$-Vollständigkeit}}
        Eine Sprache $L$ ist $\Sigma^p_i$-vollständig, genau dann wenn $L \in \Sigma^p_i$ und wenn gilt:
        $$
        \forall L' \in \Sigma^p_i: L' \leq_p L
        $$
    \end{block}
    \begin{block}{\textbf{$\Pi^p_i$-Vollständigkeit}}
        Eine Sprache $L$ ist $\Pi^p_i$-vollständig, genau dann wenn $L \in \Pi^p_i$ und wenn gilt:
        $$
        \forall L' \in \Pi^p_i: L' \leq_p L
        $$
    \end{block}
    
\end{frame}

\begin{frame}
    \frametitle{Kollabieren der PH}
    \framesubtitle{Vollständigkeit}
    \begin{block}{\textbf{PH-Vollständigkeit}}
        Eine Sprache $L$ ist PH-vollständig, genau dann wenn $L \in \text{PH}$ und wenn gilt:
        $$
        \forall L' \in \text{PH}: L' \leq_p L
        $$
    \end{block}
    
    \begin{conjecture}
        Es gibt keine PH-vollständige Sprache.
    \end{conjecture}
    Die Überlegungen dahinter werden noch erläutert.
\end{frame}

\begin{frame}
    \frametitle{Kollabieren der PH}
    \framesubtitle{\textbf{TQBF}}
    
    Für die Erläuterungen der vollständigen Mengen der PH benötigen wir die Definition des Problems \textbf{TQBF}:
      \begin{block}{\textbf{\textbf{TQBF}}}

        \textbf{Eingabe:} Eine vollständig quantifizierte boolsche Formel mit boolscher Formel $\phi$ $Q_1u_1Q_2u_2...\phi(u_1, u_2, ...)$ mit $Q_i \in \{\exists, \forall\}\\ 
        \textbf{Frage:} Gibt es eine Belegung der $u_i$, sodass $Q_1u_1Q_2u_2...$\phi(u_1, u_2, ...)$ wahr ist?

    \end{block}
    
    \begin{block}{\textbf{Satz}}
        \textbf{TQBF} ist PSPACE-vollständig (ohne Beweis).
    \end{block}
    \textbf{TQBF} kann als eine Verallgemeinerung von SAT angesehen werden, sodass SAT ein Spezialfall von \textbf{TQBF} ist mit ausschließlich existenziellem Quantor.
    
\end{frame}

\begin{frame}
    \frametitle{Kollabieren der PH}
    \framesubtitle{Vollständige Mengen}
    \begin{block}{\textbf{$\Sigma_i$-SAT}}
        \textbf{Eingabe:} Eine quantifizierte boolsche Formel mit boolscher Formel $\phi$ $\exists u_1 \forall u_2...\phi(u_1, u_2, ...)$  und höchstens $i - 1$ Alternierungen der Quantoren\\ 
        \textbf{Frage:} Gibt es eine Belegung der $u_i$, sodass $\exists u_1 \forall u_2...\phi(u_1, u_2, ...)$ wahr ist?

    \end{block}
    
     \begin{block}{\textbf{$\Pi_i$-SAT}}
        \textbf{Eingabe:} Eine quantifizierte boolsche Formel mit boolscher Formel $\phi$ $\forall u_1 \exists u_2...\phi(u_1, u_2, ...)$ und höchstens $i - 1$ Alternierungen der Quantoren\\ 
        \textbf{Frage:} Gibt es eine Belegung der $u_i$, sodass $\forall u_1 \exists u_2...\phi(u_1, u_2, ...)$ wahr ist?
    \end{block}
    Diese Probleme sind $\Sigma^p_i$ bzw. $\Pi^p_i$ vollständig.
\end{frame}

\begin{frame}
    \frametitle{Kollabieren der PH}
    \framesubtitle{Vollständige Mengen}
    Ein Paper von Schafer und Umans fasst eine große Anzahl an vollständigen Problemen abseits von $\Sigma_i$-SAT und $\Pi_i$-SAT ab der Stufe $2$.
    Bekannte Beispiele sind z.B.
    \begin{itemize}
        \item \textbf{MIN DNF}
        \item \textbf{GRAPH CONSISTENCY}
    \end{itemize}
\end{frame}

\begin{frame}
    \frametitle{Kollabieren der PH}
    \framesubtitle{Vollständige Mengen}
     \begin{block}{\textbf{MIN DNF}}
        \textbf{Eingabe:} Eine boolsche Formel $\phi$ in DNF und ein $k \in \mathbb{N}$. Die Größe einer Formel sei definiert als die Anzahl der Vorkomnisse der Literale in der Formel.\\ 
        \textbf{Frage:} Gibt es eine Formel in DNF $\psi$, sodass $\psi \equiv \phi$ und die Größe von $\psi$ höchstens $k$ ist?
    \end{block}
    
    \begin{block}{\textbf{GRAPH CONSISTENCY}}
        \textbf{Eingabe:} Zwei Mengen $A$ und $B$ bestehend aus Graphen.\\ 
        \textbf{Frage:} Gibt es einen Graphen $G$, sodass jeder Graph in $A$ isomorph zu einem Teilgraphen von $G$ ist, aber kein Graph in $B$ isomorph zu einem Teilgraphen von $G$ ist?
    \end{block}
    Beide Probleme sind $\Sigma^p_2$-vollständig.
\end{frame}

\subsection{Szenarien}
\begin{frame}
    \frametitle{Kollabieren der PH}
    \framesubtitle{Kollabieren}
    Es besteht die Vermutung, dass die PH über unendlich viele Stufen verfügt, und eine Stufe als echte Teilmenge in der jeweils höheren Stufe enthalten ist. 
    Wenn die PH lediglich über eine endliche Anzahl von Stufen $k$ verfügt spricht man von einem \emph{Kollabieren} der PH auf die Stufe $k$.
    Es werden nun verschiedene Szenarien beleuchtet, die genau dazu führen würden.
\end{frame}

\begin{frame}
    \frametitle{Kollabieren der PH}
    \framesubtitle{Szenario 1: Vollständige Sprache der PH}
     \begin{block}{\textbf{Satz}}
        Sei $L$ eine PH-vollständige Sprache. Dann kollabiert die PH auf eine endliche Anzahl von Stufen.
    \end{block}
    
    \begin{proof}[Beweis]
        Da $L \in \text{PH}$ gibt es ein $i$, sodass $L \in \Sigma^p_i$.
        Außerdem gilt, da $L$ PH-vollständig ist:
        $$
            \forall L' \in \text{PH}: L' \leq_p L
        $$
        Somit können alle Problem aus den Stufen $j$ mit $j > i$ auf L reduziert werden. Das bedeutet also:
        $$
        \text{PH} \subseteq \Sigma^p_i
        $$
        Da aber jede Klasse in der PH eine Teilmenge von PH ist, folgt daraus, dass PH nur über $i$ Stufen verfügt und auf die Stufe $i$ kollabiert.
    \end{proof}
\end{frame}

\begin{frame}
    \frametitle{Kollabieren der PH}
    \framesubtitle{Szenario 2: $\Sigma^p_i = \Pi^p_i$}
     \begin{block}{\textbf{Satz}}
        Sei $i > 0$, sodass $\Sigma^p_i = \Pi^p_i$ gilt. Dann kollabiert die PH auf die Stufe $i$.
    \end{block}
    
    \begin{block}{Beweis (1).}
         Sei $L \in \Sigma^p_{i+1}$. Dann gilt:
         \small
        \begin{align*}
        x \in L \Leftrightarrow \exists u_1 \forall u_2 ... Q_{i+1}u_{i + 1} : M(x, u_1, ..., u_{i+1}) = 1
        \end{align*}
        Der Teil der Formel nach $\exists u_1$ kann dabei als Sprache $L' \in \Pi^p_i$ aufgefasst werden.
        Daraus folgt:
        \small
        \begin{align*}
        x \in L \Leftrightarrow \exists u_1 \langle x, u_1 \rangle \in L'
        \end{align*}
        Nach der Annahme, dass  $\Sigma^p_i = \Pi^p_i$ gibt ist also auch $L' \in \Sigma^p_i$, womit gilt:
        \small
        \begin{align*}
        \langle x, u_1 \rangle \in L' \Leftrightarrow \exists v_1 \forall v_2 ... Q_i v_i : M'(\langle x, u_1 \rangle, v_1, ..., v_i) = 1 
        \end{align*}
    \end{block}
\end{frame}

\begin{frame}
    \frametitle{Kollabieren der PH}
    \framesubtitle{Szenario 2: $\Sigma^p_i = \Pi^p_i$}
    
    \begin{proof}[Beweis (2)]
         Eingesetzt in die Definition der Sprache $L$ ergibt sich:
         \small
        \begin{align*}
        & x \in L \Leftrightarrow \exists u_1 \langle x, u_1 \rangle \in L' \\
        & \Leftrightarrow \exists u_1 (\exists v_1, \forall v_2, ..., Q_i v_i :  M'(\langle x, u_1 \rangle, v_1, ..., v_i) = 1) \\
        & \Leftrightarrow \exists \langle u_1, v_1 \rangle \forall v_2, ..., Q_i v_i : M'(\langle x, u_1 \rangle, v_1, ..., v_i) = 1
        \end{align*}
        Somit ist $L \in \Sigma^p_i$, wurde aber ursprünglich beliebig aus $\Sigma^p_{i+1}$ gewählt, sodass nun $\Sigma^p_i = \Sigma^p_{i+1}$ gezeigt ist. Das Kollabieren für eine Stufe $k \geq i$ auf $\Sigma^p_i$ ergibt sich induktiv:
        \begin{align*}
        \Sigma^p_{i+2} = \text{NP}^{\Sigma^p_{i+1}} = \text{NP}^{\Sigma^p_i} = \Sigma^p_{i+1} = \Sigma^p_i
        \end{align*}
    \end{proof}
\end{frame}


\begin{frame}
    \frametitle{Kollabieren der PH}
    \framesubtitle{Szenario 2: $\Sigma^p_i = \Pi^p_i$}
    Interessante Folgerungen daraus:
    \begin{itemize}
        \item Falls P = NP, dann P = PH
        \item Falls NP = coNP, dann NP = PH
    \end{itemize}
    Die Tragweite der Antworten auf diese Frage übersteigt also ihre Relation untereinander.
\end{frame}


\begin{frame}
    \frametitle{Kollabieren der PH}
    \framesubtitle{Szenario 3: PH $=$ PSPACE}
     \begin{block}{\textbf{Satz}}
        Wenn PH $=$ PSPACE, dann kollabiert die PH auf eine endliche Anzahl von Stufen. 
    \end{block}
    
    \begin{proof}[Beweis]
        \textbf{TQBF} ist bereits als PSPACE-vollständiges Problem bekannt. Da nun PH $=$ PSPACE, gibt es ein $i$ sodass \textbf{TQBF} $\in \Sigma^p_i$. Somit kann jedes $L \in \text{PH}$ auf \textbf{TQBF} reduziert werden, sodass die gesamte PH in der Stufe $i$ enthalten ist, und damit nur über $i$ Stufen verfügt.
    \end{proof}
\end{frame}


\section{Fazit}

\begin{frame}
    \frametitle{Fazit}
    Die Polynomialzeithierarchie bietet einen Formalismus zur Beschreibung von Problemen, die vom Nichtdeterminismus nicht hinreichend beschrieben werden.
    Die Frage nach dem Kollaps der Hierarchie ist nicht abschließend geklärt, erscheint aber als unwahrscheinlich.
    Die Folge wäre, dass die erhöhte Berechungskraft über Alternierungen bzw. Orakel keine echte Erhöhung bewirken.
    In einem Gewisse Sinne wäre damit die Annahme, dass der Nichtdeterminismus und seine Verallgemeinerungen keine echten Mehrwert in der Berechnung bieten.
    Die Aussage P $=$ NP gilt bereits als unwahrscheinlich, deren unmittelbare Folge ein solcher Kollaps wäre.
\end{frame}


\begin{frame}[allowframebreaks]
\frametitle{Quellen}
\printbibliography[heading=none]

\end{frame}

\end{document}