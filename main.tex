\documentclass{beamer}
\usetheme{Boadilla}

\title{Polynomialzeithierarchie}
\subtitle{Seminar Komplexität}
\author{Edgar Wolf}
\institute{Hochschule Kempten}
\date{29.06.2022}

\begin{document}

\begin{frame}
    \titlepage
\end{frame}

\begin{frame}
    \frametitle{Agenda}
    \tableofcontents
\end{frame}

\section{Alternierung}

\begin{frame}
    \frametitle{Alternierung}
    Es gibt Probleme, die nicht ausschließlich nichtdeterminstisch gelöst werden können, jedoch immer noch schneller als mit exponentieller Laufzeit.
    Die Beschränkung des Nichtdeterminismus auf die Frage nach der Existenz einer akzeptierenden Berechung limitiert die Lösungsmöglichkeiten für Probleme,
    die Antworten über mehrere Stufen qunatifizieren. Die Alternierung löst dieses Problem.
\end{frame}

\begin{frame}
    Alternierung kannn als Verallgmeinerung des Nichtdeterminismus aufgefasst werden, und ist somit ebenfalls ein Konzept, dass in der Praxis kaum umzusetzen ist.
    Während beim Nichtdeterminismus nach einem akzeptierenden Berechnunspfad gesucht wird, können in der Alternierung die Berechnungen quantifiziert werden.
\end{frame}

\subsection{Begriffsklärung}
\begin{frame}
    \frametitle{Begriffsklärung}
\end{frame}

\subsection{Alternierende Turingmaschinen}
\begin{frame}
    \frametitle{Alternierende Turingmaschinen}
\end{frame}


\section{Polynomialzeithierarchie}
\subsection{Defintion über alternierende Quantoren}
\subsection{Defintion über alternierende Turingmaschinen}
\subsection{Defintion über Turingmaschinen mit Orakel}

\section{Eigenschaften der Polynomialzeithierarchie}
\subsection{Vollständige Mengen}
\subsection{Inklusionen}
\subsection{Kollabieren}



\end{document}