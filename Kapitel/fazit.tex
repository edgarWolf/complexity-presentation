\section{Fazit}

\begin{frame}
    \frametitle{Fazit}
    Die Polynomialzeithierarchie bietet einen Formalismus zur Beschreibung von Problemen, die vom Nichtdeterminismus nicht hinreichend beschrieben werden.
    Die Frage nach dem Kollaps der Hierarchie ist nicht abschließend geklärt, erscheint aber als unwahrscheinlich.
    Die Folge wäre, dass die erhöhte Berechungskraft über Alternierungen bzw. Orakel keine echte Erhöhung bewirken.
    In einem Gewisse Sinne wäre damit die Annahme, dass der Nichtdeterminismus und seine Verallgemeinerungen keine echten Mehrwert in der Berechnung bieten.
    Die Aussage P $=$ NP gilt bereits als unwahrscheinlich, deren unmittelbare Folge ein solcher Kollaps wäre.
\end{frame}


\begin{frame}[allowframebreaks]
\frametitle{Quellen}
\printbibliography[heading=none]

\end{frame}