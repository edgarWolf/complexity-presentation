\section{Einführung}

\begin{frame}
    \frametitle{Einführung}
    Der Nichtdeterminismus dient als ein Konzept, um die Berechnungskraft eines Algorithmus zu erhöhen.
    Jedoch hat auch der Nichtdeterminismus seine Grenzen:
    \begin{itemize}
        \item Probleme aus NP fragen, ob eine Berechnung akzeptiert: \\
         \enquote{Gibt es eine Belegung für die boolsche Formel $F$, sodass $F$ erfüllbar ist?}
        \item Probleme aus coNP fragen, ob alle Berechnungen akzeptieren: \\
          \enquote{Gilt für alle Belegungen für die boolsche Formel $F$, dass $F$ nicht erfüllbar ist?}
    \end{itemize}
    $\rightarrow$ Was tun mit einem Problem, dass beide Bedingungen entscheiden muss?
\end{frame}


 \begin{frame}
    \begin{block}{\textbf{EXACT-INDSET}:}
        \textbf{Eingabe:} Ein Graph $G$, ein $k \in \mathbb{N}$\\ 
        \textbf{Frage:} Hat das größte \textbf{INDSET} in $G$ genau die Größe $k$?
    \end{block}

    Das Problem scheint über kein \enquote{kurzes} Zertifikat zu verfügen. \\
    \textbf{Algorithmus zur Entscheidung des Problems}:
    \begin{itemize}
        \item Rate ein \textbf{INDSET} $U$ mit Größe $k$.
        \item Prüfe, dass es kein \textbf{INDSET} mit einer Größe $l > k$ gibt.
    \end{itemize}
     Wie können wir Probleme solcher Art formal beschreiben?
 \end{frame}
 
 \begin{frame}{Einführung}
    Es gibt Probleme, deren Entscheidbarkeit über den Einsatz nichtdeterministischer Berechnungen hinausgeht.
    Die Beschränkung des Nichtdeterminismus auf die Frage nach der Existenz einer akzeptierenden Berechnung limitiert die Lösungsmöglichkeiten für Probleme,
    die Antworten über mehrere Stufen quantifizieren. \\
    $\rightarrow$ Die Polynomialzeithierarchie bildet diese Probleme ab.
 \end{frame}