\section{Einführung}

\begin{frame}
    \frametitle{Einführung}
    Der Nichtdeterminismus dient als ein Konzept, um die Berechnungskraft eines Algorithmus zu erhöhen.
    Jedoch hat auch der Nichtdeterminismus seine Grenzen:
    \begin{itemize}
        \item Probleme aus NP fragen, ob eine Berechnung akzeptiert: \\
         \enquote{Gibt es eine Belegung für die boolsche Formel $F$, sodass $F = 1$?}
        \item Probleme aus coNP fragen, ob alle Berechnungen akzeptieren: \\
          \enquote{Gilt für alle Belegungen für die boolsche Formel $F$, dass $F = 0$?}
    \end{itemize}
    $\rightarrow$ Was tun mit einem Problem, dass beide Bedingungen entscheiden muss?
\end{frame}


 \begin{frame}
    \begin{block}{\textbf{EXACT-INDSET}:}
        \begin{tabular}{@{}l@{}cl}
        Gegeben &:& Ein Graph $G$, $k \in \mathbb{N}$\\ 
        Frage &:& Hat das größte \textbf{INDSET} in $G$ genau die Größe $k$?
        \end{tabular}
    \end{block}

    Das Problem scheint über kein \enquote{kurzes} Zertifikat zu verfügen. \\
    \textbf{Algorithmus zur Entscheidung des Problems}:
    \begin{itemize}
        \item Ermittle ein \textbf{INDSET} $U$.
        \item Prüfe für alle anderen Instanzen von \textbf{INDSET} $U'$, ob die Grö0e von $U' > k$.
    \end{itemize}
     Wie können wir Probleme solcher Art formal beschreiben?
 \end{frame}