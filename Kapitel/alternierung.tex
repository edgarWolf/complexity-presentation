\section{Alternierung}

\begin{frame}
    \frametitle{Alternierung}
    Es gibt Probleme, die nicht ausschließlich nichtdeterminstisch gelöst werden können, jedoch immer noch schneller als mit exponentieller Laufzeit.
    Die Beschränkung des Nichtdeterminismus auf die Frage nach der Existenz einer akzeptierenden Berechung limitiert die Lösungsmöglichkeiten für Probleme,
    die Antworten über mehrere Stufen qunatifizieren. Die Alternierung löst dieses Problem.
\end{frame}

\begin{frame}
    \frametitle{Alternierung}
    Alternierung kannn als Verallgmeinerung des Nichtdeterminismus aufgefasst werden, und ist somit ebenfalls ein Konzept, dass in der Praxis kaum umzusetzen ist.
    Während beim Nichtdeterminismus nach einem akzeptierenden Berechnunspfad gesucht wird, können in der Alternierung die Berechnungen quantifiziert werden.
\end{frame}

\subsection{Begriffsklärung}
\begin{frame}
    \frametitle{Alternierung}
    Die Zustände einer Berechnung einer alternierenden Berechnung sind (neben Startzustand und Endzuständen) entweder:
    \begin{itemize}
        \item existenziell: Die Berechnung akzeptiert, wenn mind. eine Berechnung akzeptiert
        \item universell: Die Berechnung akzeptiert, wenn alle Berechnungen akzeptieren.
    \end{itemize}

\end{frame}

\subsection{Alternierende Turingmaschinen}
\begin{frame}
    \frametitle{Alternierende Turingmaschinen}
\end{frame}