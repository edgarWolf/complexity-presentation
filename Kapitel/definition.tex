\section{Defintionen der PH}
\begin{frame}
    \frametitle{Polynomialzeithierarchie}
    Die Polynomialzeithierarchie ist... 
    \begin{itemize}
        \item ein Formalismus zur Verallgemeinerung des Nichtdeterminismus
        \item eine hierarchische Struktur über polynomiell zeitbeschränkte Komplexitätsklassen
        \item ist in PSPACE enthalten
        \item ein Modell zur Beschreibung von Problemen wie z.B. \texttt{EXACT-INDSET}
    \end{itemize}
    Die Definition kann dabei über verschiedene Ansätze erfolgen.
\end{frame}

\subsection{Definition über alternierende Quantoren}
\begin{frame}
    \frametitle{Polynomialzeithierarchie}
    \framesubtitle{Definition über alternierende Quantoren}
    \begin{block}{\textbf{Definition über alternierende Quantoren}}
        Eine Sprache $L$ ist in $\Sigma^p_i$ mit $i \geq 1$ und einem Polynom $p$, wenn es eine polynomiell zeitbeschränkte Turingmaschine $M$ gibt, sodass für jede Eingabe $x \in \{0, 1\}^*$ gilt:

        \small
        \begin{align*}
            x \in L \Leftrightarrow \exists u_1 \in \{0,1\}^{p(|x|)} \forall u_2 \in \{0,1\}^{p(|x|)} ... Q_i u_i \in \{0,1\}^{p(|x|)} M(x, u_1, ..., u_i) = 1
        \end{align*}
        wobei $Q_i = \exists$ wenn $i$ ungerade ist und $Q_i = \forall $ sonst \\
       Die Definition für $\Pi^p_i$ erfolgt analog zu der $\Sigma_i^p$ mit entsprechender Quantifizierung:
        \small
        \begin{align*}
            x \in L' \Leftrightarrow \forall u_1 \in \{0,1\}^{p(|x|)} \exists u_2 \in \{0,1\}^{p(|x|)} ... Q_i u_i \in \{0,1\}^{p(|x|)} M(x, u_1, ..., u_i) = 1
        \end{align*}
        Die PH ist definiert als die Vereinigung dieser Klassen:
        \begin{align*}
            \text{PH} = \bigcup_{i \geq 1} \Sigma^p_i = \bigcup_{i \geq 1} \Pi^p_i 
        \end{align*}
    \end{block}
\end{frame}

\subsection{Definition über alternierende Turingmaschinen}
\begin{frame}
    \frametitle{Polynomialzeithierarchie}
    \framesubtitle{Definition über alternierende Turingmaschinen}
    \begin{block}{\textbf{$\Sigma_i$TIME($f(n))$}}
        $\Sigma_i$TIME$f(n)$ ist die Menge aller Sprachen, die von einer $f(n)$-zeitbeschränkten alternierenden Turingmaschine mit existenziellem Startzustand mit höchstens $i-1$ Alternierungen entschieden werden kann.
    \end{block}
    \begin{block}{\textbf{$\Pi_i$TIME($f(n)$)}}
        $\Pi_i$TIME$f(n)$ ist die Menge aller Sprachen, die von einer $f(n)$-zeitbeschränkten alternierenden Turingmaschine mit universellem Startzustand mit höchstens $i-1$ Alternierungen entschieden werden kann.
    \end{block}
    
\end{frame}

\begin{frame}
    \frametitle{Polynomialzeithierarchie}
    \framesubtitle{Definition über alternierende Turingmaschinen}
    \begin{block}{\textbf{Definition über alternierende Turingmaschinen}}
        \begin{align*}
            \Sigma^p_i = \bigcup_{c \in \mathbb{N}} \Sigma_i TIME(n^c) \\
            \Pi^p_i = \bigcup_{c \in \mathbb{N}} \Pi_i TIME(n^c) \\
            \text{PH} = \bigcup_{i \geq 1} \Sigma^p_i = \bigcup_{i \geq 1} \Pi^p_i 
        \end{align*}
    \end{block}
\end{frame}

\subsection{Definition über Turingmaschinen mit Zugriff auf Orakel}
\begin{frame}
    \frametitle{Polynomialzeithierarchie}
    \framesubtitle{Orakel}
    Eine weitere Definition bedient sich der Orakel:
    \begin{block}{\textbf{Definition über Turingmaschinen mit Zugriff auf Orakel}}
        Die Polynomialzeithierarchie ist für $i \geq 0$ definiert als:
        \begin{align*}
             & \Delta^p_0 = \Sigma^p_0 = \Pi^p_0 = \text{P} \\
             & \Delta^p_{i+1} = \text{P}^{\Sigma^p_i} \\
             &\Sigma^p_{i+1} = \text{NP}^{\Sigma^p_i} \\
             & \Pi^p_{i+1} = \text{coNP}^{\Sigma^p_i}, \Pi^p_{i+1} = \text{co}\Sigma^p_{i+1} \\
             & \text{PH} = \bigcup_{i \geq 0} \Sigma^p_i
        \end{align*}
    \end{block}
\end{frame}

\begin{frame}
    \frametitle{Polynomialzeithierarchie}
    \framesubtitle{Definition über Turingmaschinen mit Zugriff auf Orakel}
    Die Definition über Orakel betrachtet eine Stufe $0$, sowie die Klasse P, die Zugriff auf Orakel hat, während
    die anderen Definitionen ihre Definition in Stufe $1$ beginnen. Für die strukturelle Betrachtung der PH und bei der Frage des Kollapses spielen aber die Klassen der $\Delta^p_i$ eine nebensächliche Rolle.
\end{frame}

